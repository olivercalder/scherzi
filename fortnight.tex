\documentclass{article}
\usepackage[utf8]{inputenc}
\usepackage{amsmath, amssymb, amsthm}

\title{Problems of the Fortnight}
\author{Oliver Calder}
\date{11 March 2022}

\begin{document}

\maketitle

\noindent 2. Consider an $n \times n$ matrix (square array), all of whose entries are 0 and 1, for example
\begin{align*}
    \begin{tabular}{cccc}
        1 & 1 & 0 & 1 \\
        1 & 0 & 0 & 1 \\
        0 & 0 & 0 & 1 \\
        0 & 0 & 0 & 0
    \end{tabular}.
\end{align*}
If we look at the $n$ rows of this matrix and interpret them as binary representations of integers (in the example, $n = 4$, and the integers are $1101_2 = 8 + 4 + 1 = 13$, $1101_2 = 9$, $0001_2 = 1$, $0000_2 = 0$), then add up those integers, we get an integer $R$ (in the example, $R = 13 + 9 + 1 + 0 = 23$).
If, instead, we interpret the \textit{columns} of the matrix as binary numbers and we add those numbers, we get another integer $C$ (in the example, the columns yield the integers $1100_2 = 12$, $1000_2 = 8$, $0000_2 = 0$, $1110_2 = 14$, and so $C = 12 + 8 + 0 + 14 = 34$).
Finally, we can subtract $R$ from $C$ to get a single integer $D = C - R$ (in the example, $D = 11$).
Now for the problem:
\begin{enumerate}
    \item[a)] As a function of $n$, what is the largest possible value of $D$ (for all possible $n \times n$ matrices with entries chosen from \{0, 1\})?

        \vspace{10pt}

        Let $A$ denote the $n \times n$ matrix of the form described, and let $a_{i, j}$ denote the value of the entry in the $i^{th}$ row and the $j^{th}$ column of $A$.
        Note that if $a_{i, j} = 1$, then it contributes $2^{n - i}$ to the total value of $C$ and it contributes $2^{n - j}$ to the total value of $R$.
        Thus, \[ D = C - R = \sum_{i = 1}^n \sum_{j = 1}^n a_{i, j} \left( 2^{n - i} - 2^{n - j} \right) . \]
        Since we want to maximize $D$, we want to include entries where the former term in the difference is greater than the latter term, and thus we want that $j > i$.
        When $i = j$, the entry contributes equally to both $C$ and $R$, so it does not matter whether $a_{i, i}$ is 0 or 1 (see part b) for further discussion of this point).
        Thus, the matrix with the highest value of $D = C - R$ is the matrix with 1s in all entries above the diagonal and 0s in all entries below the diagonal.
        For simplicity, let all entries on the diagonal be 0 as well (though any of them could be 1 with no effect on the overall value of $D$).
        Thus, in the $n = 4$ case, the highest-valued matrix is
        \begin{align*}
            A =
            \begin{tabular}{cccc}
                0 & 1 & 1 & 1 \\
                0 & 0 & 1 & 1 \\
                0 & 0 & 0 & 1 \\
                0 & 0 & 0 & 0
            \end{tabular}.
        \end{align*}
        The transpose of the matrix gives an equivalent way to read the columns as binary numbers.
        \begin{align*}
            A^\top
            = \begin{tabular}{cccc}
                0 & 1 & 1 & 1 \\
                0 & 0 & 1 & 1 \\
                0 & 0 & 0 & 1 \\
                0 & 0 & 0 & 0
            \end{tabular}^\top
            = \begin{tabular}{cccc}
                0 & 0 & 0 & 0 \\
                1 & 0 & 0 & 0 \\
                1 & 1 & 0 & 0 \\
                1 & 1 & 1 & 0
            \end{tabular}
        \end{align*}
        Thus, with some abuse of notation, we can view the maximum value of $D = C - R$ as the sum of differences across the rows of $A^\top$ and $A$.
        \begin{align*}
            D_{max} &= A^\top -_{row} A \\
            &= \begin{tabular}{cccc}
                0 & 0 & 0 & 0 \\
                1 & 0 & 0 & 0 \\
                1 & 1 & 0 & 0 \\
                1 & 1 & 1 & 0
            \end{tabular}
            - \begin{tabular}{cccc}
                0 & 1 & 1 & 1 \\
                0 & 0 & 1 & 1 \\
                0 & 0 & 0 & 1 \\
                0 & 0 & 0 & 0
            \end{tabular} \\
            &= \begin{tabular}{cccc}
                1 & 0 & 0 & 0 \\
                1 & 1 & 0 & 0 \\
                1 & 1 & 1 & 0 \\
                0 & 0 & 0 & 0
            \end{tabular}
            - \begin{tabular}{cccc}
                0 & 1 & 1 & 1 \\
                0 & 0 & 1 & 1 \\
                0 & 0 & 0 & 1 \\
                0 & 0 & 0 & 0
            \end{tabular} \\
            &= \begin{tabular}{cccc}
                1 & 1 & 1 & 1 \\
                1 & 1 & 1 & 1 \\
                1 & 1 & 1 & 1 \\
                0 & 0 & 0 & 0
            \end{tabular}
            - 2 \left( \begin{tabular}{cccc}
                0 & 1 & 1 & 1 \\
                0 & 0 & 1 & 1 \\
                0 & 0 & 0 & 1 \\
                0 & 0 & 0 & 0
            \end{tabular} \right) \\
            &= (n - 1) (2^n - 1) - 2 \sum_{k = 1}^{n - 1} 2^k - 1 \\
            &= (n - 1) (2^n - 1) - \sum_{k = 1}^{n - 1} 2^{k + 1} - 2 \\
            &= (n - 1) (2^n - 1) + 2(n - 1) - \sum_{k = 1}^{n - 1} 2^{k + 1} \\
            &= (n - 1) (2^n + 1) - \sum_{k = 2}^{n} 2^{k}
        \end{align*}

    \item[b)] Still as a function of $n$, how many different values of $D$ can occur?
        How do you know?
\end{enumerate}


\end{document}
